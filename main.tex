\documentclass[uplatex]{jsarticle}

\title{%
  インドネシア語入門 \\
  \large 論理的なアプローチ}
\author{Agro Rachmatullah}

\begin{document}

\maketitle

\tableofcontents

\section{はじめに}

インドネシア語の教科書はたくさんありますが、ほとんどが以下の問題を持っています。

\begin{enumerate}
  \item できるだけ早く学習者に簡単な会話をしてもらうように、フレーズ本のようなものになっていること
  \item 最初から読者を驚かさないように、正確な発音の説明を省いてカナ発音でごまかすこと
  \item 会話で実際に使われているインドネシア語、いわゆるくだけたインドネシア語を教えないこと
\end{enumerate}

インドネシア語と日本語の文法はだいぶ違うので、文章を作るときの文の組み立て方ももちろん違います。例えば、友達の近況を聞く時に、日本語では「元気?」というひと言で済みますが、インドネシア語では「Sehat-sehat aja?」という言い方があって、直訳すると「元気元気だけ?」となって面白おかしく感じられますが、インドネシア語の文法によってこのフレーズの構造を説明することができます。フレーズ本のような教科書で勉強すると早く簡単な会話ができるかもしれませんが、場面ごとの複雑なパターンをたくさん丸暗記しないといけなくなります。学習者は結局合理的な文法の理解が把握できないままで、複雑な考えを正確に正しくインドネシア語で伝えるのは困難です。

その問題の解決法は文法を簡単なものからだんだん複雑なものへと順序良く教えることです。文法を1つずつゆっくり取り込むことで、その背景にあるインドネシア人の考えも十分説明でき、習得した知識が次の文法の学習や複雑な文章の作成の基礎となります。

この教科書はインドネシア語の文法を深く理解できるように文法の知識を積み上げ式で築いていくというアプローチを採用しています。つまり[B]という文法を理解するには[A]という文法が必要なら、ある便利なフレーズを早く教えたいからといって先に[B]を教えるようなことは避けています。そのため、順番や説明は他の教科書とだいぶ違うかもれませんし、会話ができるまで時間もかかるかもしれませんが、順調にレッスンを進むことができると確信しています。

インドネシア語は日本語と同じように1つの単語が変わるだけで異なるニュアンスを持つ文章になります。著者はインドネシア語を母語として日本語を勉強しているインドネシア人ですが、新しい文法を紹介する際にインドネシア人がその文法を用いた文章を使う時にどう考えたり感じたりするかをできるだけ詳しく説明して、同じような意味の日本語と比べてみます。

外国語で会話をするときに、伝えたいことを相手に理解してもらうことが一番大事です。もちろん正しいインドネシア語の文章を作るのは言うまでもなく必要ですが、文章に問題がなくても発音が綺麗ではないと聞き取りにくいです。インドネシア語はローマ字で書かれているのでそのままカタカナみたいに読みがちですが、間違った発音になる場合が多いです。そのためこの教科書はしっかり発音できるようにそれぞれの音をきちんと習ってもらいます。

最後に、従来の教科書で習うインドネシア語はどちらかというとフォーマルなインドネシア語です。ニュース、新聞、スピーチ、講義、論文などは教科書のインドネシア語を使いますが、実際の会話で使われているインドネシア語は単語も動詞の変化などの文法も違います。ここでそのインドネシア語の一種を会話インドネシア語と言って、従来教科書で教えてきたインドネシア語を教科書インドネシア語と言います。元々インドネシア語はそれぞれの地域言語を語る種族を結合させるために作られたので、公式な文法から外れる会話インドネシア語はみんなに理解してもらえないからといって、間違っているインドネシア語で使うべきではないと先生や教科書がその存在自体を否定してきましたが、言語というのは進化するもので今はそのくだけたインドネシア語がもう広く普及されていて実際の会話で使われているので、会話のときに教科書インドネシア語を使うとぶっちゃけ言うと不自然です。

インドネシア人の子供は親、友達、テレビから言語を習って自然に会話インドネシア語を見につけたけど、現在のインドネシア人向けの国語の教科書の目的はそのくだけたインドネシア語を直すためだからもちろん会話インドネシア語は国語の教科書に載っていません。しかし、その場にふさわしい言葉が言えるのは言語学習の目的だから、その観点から見れば会話の時は教科書インドネシア語は実は間違っていて、正しいのは会話インドネシア語です。従って、インドネシア語を第二言語とした学習者には会話インドネシア語を教えるのは必要不可欠です。

教科書インドネシア語は使われていないわけではないし、会話インドネシア語の文法の基礎となるので、会話インドネシア語の説明は教科書インドネシア語の後にきます。

\section{なぜインドネシア語を勉強するか}

\section{言語を勉強するためのアドバイス}

\section{発音編}

\subsection{文字、音節、発音記号}

インドネシア語はラテン文字(ローマ字)で書かれています。英語みたいにそれは26字の大文字と小文字からなっています:

大文字: ABCDEFGHIJKLMNOPQRSTUVWXYZ

小文字: abcdefghijklmnopqrstuvwxyz

使い分けも英語と同じで、文章の始めや人の名前の頭文字を大文字で書きます。

日本語のかなは音節を表現します。例えば「あ」でも「ば」でもそのまま発音できます。一方、インドネシア語の音節は子音と母音の組合せからなっています。

インドネシア語の母音文字は5個(aiueo)あって、単独でも音節になっていますので発音できます。例えば、「a」自体は日本語の「あ」と同じく発音します。

「b」など残りの文字は子音で、母音と合わせて初めて音節として発音できます。一番簡単な構造は「子音 + 母音」です。例えば「b」という子音と「u」という母音を合わせたら「bu」という音節になって日本語の「ぶ」と発音します。

カタカナで表現できるインドネシア語の音は限られているので、ここで大括弧の中のローマ字で発音を示します。「·」で音節を区切って、例えば「abu」(灰)の発音は[a·bu]と書きます。

「母音 + 子音」の音節もあって、例としては「an」です。「anda」(あなた)の発音は「あんだ」みたいで、発音記号で[an·da]と表します。

「子音 + 母音 + 子音」や「子音 + 子音 + 母音 + 子音」などもっと複雑な音節などもありますけど、後で少しずつ紹介していきます。

韓国語などを勉強する場合、その言語の文字をまず勉強しないと全然読めないです。しかしインドネシア語は見慣れたラテン文字で書かれているから一応日本語のローマ字表記のように読めます。説明したように、正しい発音と違うケースが多いですが、発音の勉強は結構大変なことで時間がかかるので、発音編を終えてから文法編に入るより、ローマ字発音で文法の勉強を始めて、同時に少しずつ発音編を読みながら自分の発音を直していくのが一番ラクかもしれません。

\subsection{aiueo}

母音は音節の基礎になるので、まず全部しっかりと習います。インドネシア語の母音文字は「aiueo」と5個あって、でも実は「e」という文字には二通り発音があるので、6個の母音があります。

「a」、「i」、と「o」の発音は日本語の「あ」、「い」、と「お」と同じなので、問題ないです。

この3つの母音からでももう単語が作れます。しかし、母音が連続で出てくると、単語によって発音も色々です。それは単語ごとに憶えるしかないですけど、ここでその違いは発音の表記で表します。

「oi」という単語は発音的にも意味的にも日本語の「おい!」と相当します。この場合は一つの音節としてoとiの音を合わせて速く発音します。発音記号で[oi]と書きます。

「多い」みたいな2音節の発音は[o·i]と書きますが、「·」は音節の区切りを表します。

「ia」(彼、彼女)はちゃんと二つの音節として発音するので、発音表記は[i·a]。「威圧」を、最後の「つ」を除いて発音するのと同じです。

「ia」の例では、実はiとaの音をスムーズにつないで「いや」と発音します。最後のパターンは切り離して発音するパターンです、例として「AI」という単語を使います。

「AI」は英語の「Artificial Intelligence(人工知能)」の略なので大文字で書きます。発音は「愛」みたいではなくて「あ・い・う・え・お」の「あ・い」の部分みたいです。これは発音記号で[a·'i]と書きます。「a」の前の「'」はつなげないでという意味です。

その3種類をまとめて下の表を書きますが、それぞれの発音の違いはオーディオファイルを聞いてください。

[ai]		一音節として発音する

[a·i]		二音節、スムーズに発音する

[a·'i]		二音節、切り離して発音する

\section{等しさ}

あるものが何かであることを伝えるのはどの言語でも一番基本的なことです。例えば、日本語では指で指すものがペンであることを伝えたいときに「これはペンだ」と言います。インドネシア語では「adalah」を使います。

\textbf{「adalah」を使ってあるものが何かだと断言する}
名詞の間に「adalah」を入れる
例: Ini \textbf{adalah} pulpen. ― これはペン\textbf{だ}。

ここで「adalah」は日本語の「だ」の役割を果たしますが、使い方は数学の「=」みたいで真ん中に入れます。「adalah」の左側と右側は同じものなので等しさの表現と言って、「A adalah B」を見たら「A = B」を想像すればいいです。

例:
Itu \textbf{adalah} buku.
あれは本\textbf{だ}。

Dia \textbf{adalah} teman.
彼は友達\textbf{だ}。

Anto \textbf{adalah} guru.
アントは先生\textbf{だ}。

簡単でしょう。でも実は「adalah」について一つ重要な使い方がある。

等しさの表現で「adalah」を省略してもいい!

これは日本語みたいに「だ」を省略して「これはペン」と言っても意味が伝わります。普通の会話では「adalah」はわざわざ言わないです。

例:
Ini pulpen.
これはペン。

Dia teman.
彼は友達。

じゃいつ「adalah」を使うのと聞くかもしれませんが、百科事典や論文などで定義をする時に「adalah」を使います。そういうような場合しか普通使わないので「adalah」は堅苦しく聞こえます。その他に「adalah」の左側または右側が長いとき、相手がその区切りが簡単にわかるように「adalah」を使う場合が多いです。

\textbf{「adalah」の否定形}

否定するときに「adalah」は「bukan」に変わります。

\textbf{「bukan」を使ってあるものが何かじゃないと否定する時}
名詞の間に「bukan」を入れる
例: Ini \textbf{bukan} pulpen. ― これはペン\textbf{じゃない}。

例:
Dia \textbf{bukan} murid.
彼は学生\textbf{じゃない}。

Itu \textbf{bukan} makanan.
あれは食べ物\textbf{じゃない}。

ご覧のように使い方は数学の「≠」みたいです。

日本語の場合「だ」の活用を説明するとき過去形の「だった」と「じゃなかった」を説明するのは当然ですが、インドネシア語では過去形の活用がないです。過去を表すために時に関する単語を入れますが、それは他の章で詳しく説明します。

\textbf{まとめ}

ここで等しさの表現とその否定形を勉強しました。次はものの性質を形容詞で表すことを勉強します。この章で勉強したことを表でまとめます。

		等しさの表現
肯定的	ini \textbf{(adalah)} pulpen
否定的	ini \textbf{bukan} pulpen

-------


あるものがなにかであること


状態
あるものと他のものが同じだということを伝えるのはどの言語でも基本的なことです。と言う時に、指したものは

---

\section{性質}

ものの性質を表すときに形容詞を使いますが、日本語で言うと「寿司は美味しい」の「美味しい」はその形容詞です。インドネシア語では名詞と形容詞をちゃんと区別して「adalah」ではなく「bersifat」を使います。

\textbf{「bersifat」を使ってあるものの性質を伝える}
説明するものと形容詞の間に「bersifat」を入れる
例: Dia \textbf{bersifat} baik.  ― 彼は優しい。

「baik」は直訳すると「いい」になりますが、<link:internal>人</link:internal>の性格として「優しい」にも訳せるでしょう。

「adalah」のように、「bersifat」も普通省略します。

例:
Dia baik.
彼は優しい。

Saya sehat.
私は健康です。

Sushi enak。
寿司は美味しい。

\textbf{「bukan」の形容詞バージョン}

ものの性質を否定する時に名詞用の「bukan」を使わないように注意してください。形容詞の場合は「tidak」を使います。

\textbf{「tidak」を使ってあるものの性質を否定する}
説明するものと形容詞の間に「tidak」を入れる
例: Dia \textbf{tidak} baik.  ― 彼は優しく\textbf{ない}。

例:
Saya \textbf{tidak} sehat.
私は健康\textbf{じゃない}。

Sushi \textbf{tidak} enak.
寿司は美味しく\textbf{ない}。

\textbf{まとめ}

名詞か形容詞を扱うことによって必要な単語も異なりますが、それだけ気をつければ難しい活用がないので簡単です。

		性質の表現
肯定的	ini \textbf{(bersifat)} enak
否定的	ini \textbf{tidak} enak

\section{形容詞で修飾する}

日本語では名詞を修飾するときに「白い雲」のように描写する単語(白い)を前に置きます。それはあたり前だと思うかもしれませんが、実はインドネシア語ではその順番は逆です!

\textbf{形容詞で修飾する}
形容詞を名詞の後に置く
例:masakan \textbf{enak} ― \textbf{おいしい}料理

上記では説明される名詞「masakan」(料理)は最初にきて、後に続くのは描写する形容詞「enak」(おいしい)です。他の例:

baju biru
青い服

pemandangan indah
きれいな景色

orang pendiam
静かな人

awan putih
白い雲

英語も日本語と同じ順番なので最初はインドネシア語の順番に違和感を感じるかもしれませんが、練習を繰り返して馴れるしかないです。もっと複雑な例:

Ini adalah mobil mahal.
これは高い車だ。

Dia anak pintar.
彼は頭がいい子だ。

Itu bukan baju baru.
あれは新しい服じゃない。

前章で説明された通り、「bersifat」は省略するので、1つの文章が曖昧な場合があります。例えば「mawar merah」は「赤いバラ」という意味にもなりますが、もしかしてそれは「mawar (bersifat) merah」で「バラは赤い」という意味の文章です。このとき文脈で正しい意味を把握するしかないですが、ほとんどの場合そらははっきりなので全然難しくないです。他の例:

komputer murah
安いパソコン
または
パソコンは安い (komputer bersifat murahの省略)

Ini komputer murah.
これは安いパソコンだ。

\textbf{yang}

形容詞の否定形は「tidak」という単語を使うので、「tidak enak」のように2単語の句になります。一単語じゃなくて、複雑な句で修飾したい場合によく「yang」を使いますが、否定形の形容詞の場合もそうです。

\textbf{否定形の形容詞で修飾する}
名詞と形容詞の間に「yang」を入れる
例:masakan yang tidak enak ― おいしくない料理

この「yang」は実は日本語の「方」と近い意味を持っているので、ニュアンス的には「おいしくない方の料理」と近いです。

他の例:
kelas yang tidak susah
難しくない授業

sungai yang tidak bersih
嫌いじゃない川

Mereka murid yang tidak rajin.
彼らは真面目じゃない学生だ。

Itu contoh yang tidak benar.
それは正しくない例だ。

\textbf{まとめ}

形容詞で修飾する時に、語順さえおぼえれば問題がないはずです。

		形容詞の修飾
肯定的	masakan enak
否定的	masakan yang tidak enak

--------------------

\textbf{iniとitu}





masakan yang enak
masakan yang tidak enak
anak yang pintar
anak yang tidak pintar

peserta yang orang jepang
peserta yang bukan orang jepang harus mendaftar
peserta bukan orang jepang harus mendaftar

buku yang bahasa indonesia
saya orang yang indonesia

peserta orang jepang tinggi
peserta orang jepang yang tinggi
peserta bukan orang jepang yang tinggi
peserta yang orang jepang dan yang tinggi

mas-mas yang mobilnya mahal itu
車が高いお兄さん

形容詞



所有

ini
itu

rumah makan
rumah sakit
buku tulis ノート
tanda tanya

orang indonesia
bahasa jepang
kamus bahasa Inggris
buku ini

orang kaya
buku biru

buku bahasa indonesia biru milik toni
トニーの青いインドネシア語の本

生涯学習センター

\section{所有}

buku milik saya

milikは日本語の「の」

省略できる

buku saya

代名詞の省略

buku aku→bukuku

buku dia→bukunya (dia)

\section{misc}

インドネシア語の文法

indonesiagonyuumon.com

\textbf{アグロのインドネシア語入門}


このサイトへようこそ。ここの資料を使ってみなさんが自分の家から無料でインドネシア語を勉強することができます。

--------------------
それぞれの文字の名前は以下のビデオでご覧できます
[ビデオ]

a ア a
b ベ bé
c チェ cé
d デ dé
e エ é
f エフ éf
g ゲ gé
h ハ ha
i イ i
j ジェ jé
k カ ka
l エル él
m エム ém
n エン én
o オ o
p ペ pé
q キ ki
r エル ér
s エス és
t テ té
u ウ u
v フェ fé
w ウェ wé
x エクス éks
y イェ yé
z ゼット zét

教科書インドネシア語編

会話インドネシア語編


フレーズを学ぶのではなく、文法を


-----------------------

b,c,d

baca
coba
dada

-----------------------

u

babu
dadu
cuci

-----------------------

n,m

nama [na-ma]
mana [ma-na]
bumi [bu-mi]
damai [da-mai]
mau [ma-u]

-----------------------

e

lele
emu

-----------------------

勉強の理由
仕事
結婚
友達
趣味

lとrの区別
pasal pasar

é-nak

rokokの最後のkは発音しない
nenek
ko'

oa
doa do-'a
soal so-'al
oasis o-a-sis
koala ko-a-la

ai
pantai pan-tai
air a-ir
santai san-tai
lain la-in

au
mau ma-u
kalau ka-lau

ng
sangat sa-ngat
lengah le-ngah
mangga mang-ga

lemahなどのhはバッハ

lain vs menggapaiのpai
表記が同じなのに発音が違う。なので、発音の表記も必要。

不規則な発音
tahu -> tau
silakan -> silahkan
menteri ->mentri

文字と発音
aiueo
eの2つの発音
abcdefghijklmnop
q = k
r
stu
v=f
w
x = s
yz

古いスペル

数字の個数(ekor, buah, lembar)

自己紹介
Perkenalkan
Saya NAMA
Saya X tahun
Saya mahasiswa
Saya orang NEGARA

慣用表現
selamat pagi
selamat siang
selamat malam
terima kasih
sama-sama
maaf
tidak apa-apa

---------------------------

教科書の文法 対 会話の文法

eh, besok ada pr nggak?
wah, aku juga lupa. ntar aku coba tanya si andi ya?
ok deh. kabarin lagi ya nanti.



eh, kamu udah denger belum kalau joni mau ke luar negeri?
yang bener? mau ke mana?
ke jepang katanya buat sekolah.
keren banget! emang dia dapet beasiswa ya?
iya katanya. Ntar di kelas kita tanya-tanya yuk.

---------------------------

教科書文法編
 
イ文→等しさ
イ文→性質
イ分→形容詞で修飾する
イ分→所有



過去
Ini dulu adalah buku
Dia dulu adalah teman
Anto dulu adalah guru
位置
Dulu Anto adalah guru
Anto adalah guru dulu
(adalah guru) はVOのペアで入れることができない

Ini dulu bukan buku
Dia dulu bukan teman
Anto dulu bukan guru

duluだけじゃなくてkemarin

Ini kemarin bukan buku
Dia tahun lalu bukan teman
Anto besok bukan guru

書き言葉

--------------------------------

ini itu
dia

Rumah ini
Rumah itu
Buku ini
Buku itu

toko nasi goreng ayam enak itu

Ini apa?
Itu apa?
Dia siapa?
Dia presiden
Dia siapa?
Dia anto

--------------------------------

Baju ini bersifat mahal
Rumah itu bersifat biru

Nasi goreng enak
Baju ini mahal
Rumah itu biru

Nasi goreng tidak enak
Baju ini tidak mahal
Rumah itu tidak biru
Dia tidak baik

Nasi goreng dulu tidak enak
...


(adalah) time marker (+ adalah)
bukan time marker + bukan

(bersifat) time marker (+bersifat)
tidak time marker (+bersifat)

---------------

harga buku itu mahal
harganya mahal
その本の値段が高い
値段が高い

harga buku mahal itu...
ituの位置が大事

cerita film itu menarik
ceritanya menarik
その映画のストーリーが面白い
ストーリーが面白い

nyaは「何か」を明記的に表す単語

---------------

buku yang harganya mahal
値段が高い本
baju yang warnanya biru
色が青い服

----------------

形容詞

Buku biru
anak baik

動詞

辞書形 = 命令形
makan!
lari!
lihat!

-------------

dia makan ikan
dia berlari
dia melihat film
dia membaca buku

ikan berenang
orang berjalan
kuda berlari

ikan tidak berenang
orang tidak berjalan

-----------------

intonation
CAPS = high
namaNYA siAPA?
namaNYA SIapa?
両方あり

------------

音の変化
bolpoin → bolpen
pantai → pante
santai → sante


\end{document}
